%%
%% Generated by gpt_translate from test/kwantoren.tex, on 2024-07-09 16:44:46 using model gpt-3.5-turbo-16k
%%

% GPT CHUNK%
\documentclass{ximera}
\input{../preamble.tex}
\addPrintStyle{../syllabusYongStyle}

\begin{document}
\author{Wim Obbels}
\title{Kwantoren}

\maketitle

\memo[Algemeen]{
De betekenis van sommige uitspraken hangt af van de context: de betekenis van de zin "hij is president van Amerika" wordt bepaald door welke 'hij' precies wordt bedoeld. In de wiskunde hangt de waarheidswaarde van een uitspraak als "$n$ is even" af van welke $n$ precies wordt bedoeld.

Uitspraken met algemene verwijzingen ('hij', 'die', 'het', '$n$', '$x$', \ldots) noemen we \textit{predicaten}. In de wiskunde noemen we de letters ('variabelen' of 'parameters') die voorkomen in dergelijk uitspraak of formule 'vrije variabelen'.
Beschouw bijvoorbeeld de uitspraken
\begin{multicols}{3} 
\begin{enumerate}
	\item $n^2 = 4$
	\item $n^2=-4$
	\item $n^2 > 0$
\end{enumerate}
\end{multicols}
Dan is de uitspraak (a) $n^2=4$ enkel waar als $n=2$ of $n=-2$, de uitpraak (b) $n^2=-4$ is voor geen enkel natuurlijk getal $n$ waar (maar wel voor de complexe getallen $\pm2i$), en uitspraak (c) $n^2>0$ is waar voor alle natuurlijke getallen $n$ (maar niet voor bijvoorbeeld het complexe getal $i$, want $i^2=-1<0$).

Er zijn essentieel drie verschillende manieren om een dergelijke 'vrije variabele $n$' te 'binden', dus vast te leggen. We kunnen inderdaad bedoelen dat de uitspraak al dan niet moet gelden voor
\begin{itemize}
	\item één \textit{concrete} waarde: $n=2$, of $n=6$
	\item \textit{alle} mogelijke waarden van $n$ (eventueel enkel van een bepaald type)
	\item \textit{minstens één} waarde van $n$ (eventueel van een bepaald type)
\end{itemize}
De beweringen worden pas waar of vals zodra dergelijke afspraak is gemaakt.
De twee laatste gevallen kunnen we uitdrukken met ``voor alle'' en ``er bestaat'' , en ze geven aan \textit{voor hoeveel} waarden (dus voor welke \textit{kwantiteit} van waarden) de uitspraak zou moeten gelden: voor allemaal of voor minstens één. Men noemt ze in de logica daarom \textbf{kwantoren} en men voert volgende notaties in:
\begin{definition}[Kwantoren]
	Als $p(a)$ een predicaat is dat afhangt van een veranderlijke $a$, en $A$ een verzameling, dan betekent
	\\
	$$
	\begin{array}{ll}
		\important{\forall a \in A : p(a)}  & \textit{voor elk} \text{ element $a$ van $A$ geldt $p(a)$} \\
        \\
		\important{\exists a \in A :p(a)}   & \textit{er bestaat (minstens) een}\text{ element $a$ van $A$ waarvoor $p(a)$ geldt} \\
        \\
		\important{\exists! a \in A : p(a)} & \textit{er bestaat precies één}\text{ element $a$ van $A$ waarvoor $p(a)$ geldt}
	\end{array}
	$$
	\\
	Hierbij noemen we 
	$$
	\begin{array}{cl}
		\forall  & \text{ de }\textbf{universele kwantor}\text{ of }\textbf{alkwantor}          \\
		\exists  & \text{ de }\textbf{existentiële kwantor}\text{ of }\textbf{bestaanskwantor} \\
		\exists! & \text{ de }\textbf{uniciteitskwantor}
	\end{array}
	$$
    Als de verzameling $A$ duidelijk blijkt uit de context wordt ze soms weggelaten, en schrijven we ook
    $$
    \begin{array}{lll}
      \forall a: p(a)  & \exists a: p(a) & \exists! x : p(x) 
    \end{array}
    $$
    of
    $$
    \begin{array}{lll}
      (\forall a) p(a)  & (\exists a) p(a) & (\exists! x) p(x) 
    \end{array}
    $$
\end{definition}

\begin{remark}\nl
\begin{itemize}
\item We schrijven de kwantor in echte formules altijd v\'o\'or het predicaat, en dus niet $\cancel{p(a),\forall a\in A}$. Maar, in een meer informele context schrijven we soms toch uitdrukkingen als
$$
x>0 \implies |x|=x \quad\quad(\forall x\in\R)
$$
\item In sommige handboeken schrijft men \textit{altijd} haakjes rond een kwantor $(\forall x): p(x)$, en dan laat men de dubbele punt soms weg $(\forall x)p(x)$. Sommigen gebruiken een komma: $\forall x,p(x)$
\item De \textit{naam} van de variabele heeft geen belang: de uitspraken $\forall a \in A : p(a)$ en $\forall x \in A : p(x)$ betekenen precies hetzelfde, en we noemen dergelijke veranderlijke een \textbf{gebonden veranderlijke}, of soms ook een \textbf{dummy} veranderlijke. 
%
% todo: onduidelijk geformuleerd; afspraken haakjes in logische formules verduidelijken !
\item Meestal spreekt men af dat een kwantor betrekking heeft op \textit{alles wat er na komt}, tenzij het door haakjes anders wordt bepaald. Dus
$$
\forall a \in A : p(a) \Rightarrow q  
% \quad\text{ of }\quad (\forall x \in A) : p(x) \Rightarrow q \quad\text{ of }\quad (\forall y \in A) : (p(y) \Rightarrow q)
$$ 
betekent $\forall a \in A : (p(a) \Rightarrow q)$, en dus dat voor elk een willekeurig element $a \in A$ geldt: zodra $p(a)$ waar is, is ook $q$ waar. Concreet zal $q$ dus zeker waar zijn zodra er minstens één $a$ is waarvoor $p(a)$ geldt.
Maar
$$
(\;\forall a \in A : p(a)\;) \Rightarrow q. 
$$
betekent dat pas als $p(a)$ geldt voor elke $a \in A$, dan ook $q$ waar is. Dat is dus een veel \textit{sterkere} voorwaarde voor dezelfde conclusie $q$, en het is dus een \textit{zwakkere} uitspraak dan $\forall a \in A : (p(a) \Rightarrow q)$.
\end{itemize}
\end{remark}
Met deze notaties kunnen we nu compact schrijven 
\begin{itemize}
	\item $\exists n\in\N: n^2=4$ voor 'er bestaat een natuurlijk getal waarvan het kwadraat gelijk is aan $4$'. (waar)
	\item $\forall n\in\N: n^2=4$ voor 'alle natuurlijke getallen hebben als kwadraat $4$'. (vals)
	\item $\forall n\in\N: n^2>0$ voor 'alle natuurlijke getallen hebben een strikt positief kwadraat'. (waar)
\end{itemize}
Het is belangrijk dat je deze uitdrukkingen met kwantoren spontaan leert lezen als volwaardige zinnen.

\section{Negatie van kwantoren}

Er is een interessante wisselwerking tussen kwantoren en ontkenningen: beweren dat \textit{niet iedereen} geslaagd is, betekent dat er \textit{iemand is} die niet geslaagd is. En beweren dat \textit{iedereen} iets \textit{niet} heeft komt op hetzelfde neer als beweren dat \textit{niemand} dat ding \textit{wel} heeft. 

\begin{example}
Als $X$ de verzameling is van alle studenten die in deze cursus hebben gelezen en $p(x)$ is de bewering ``student $x$ heeft blond haar'', dan lezen we de uitspraak 
$$
\forall x\in X: p(x)
$$ 
als ``alle studenten die in deze cursus hebben gelezen zijn blond''.

Deze uitspraak kan in de praktijk zowel waar zijn als vals, maar ze is uiteraard \textit{niet waar} van zodra er minstens één student is die niet blond is. In symbolen uitgedrukt betekent dit 
$$
\neg(\forall x\in X: p(x))\quad\Longleftrightarrow\quad \exists x\in X: \neg p(x).
$$ 
(Opdracht: lees deze uitdrukking als een gewone Nederlandse volzin.)

Op gelijkaardige manier betekent 'niemand is geslaagd' hetzelfde als 'iedereen is gebuisd'. In formules, met $q(x)$ de bewering ``$x$ is geslaagd'', wordt dit:
$$
\neg( \exists x\in X: q(x))\Longleftrightarrow \forall x\in X: \neg q(x).
$$
en 'niemand is gebuisd' betekent hetzelfde als 'iedereen is geslaagd':
$$
\neg( \exists x\in X: \neg q(x))\Longleftrightarrow \forall x\in X: q(x).
$$
(Opdracht: lees deze uitdrukkingen als gewone Nederlandse volzinnen.)
  \end{example}
Samengevat: je mag de negatie voorbij een kwantor schuiven, maar dan verandert de kwantor $\forall \leftrightarrow\exists$:

\begin{definition}[Negatie van kwantoren]
Voor elk predicaat $p(x)$ geldt:
\begin{enumerate}
	\item $\neg(\forall x\in X: p(x)) \quad\Longleftrightarrow\quad \exists x\in X: \neg p(x)$ 
    \qquad  of \qquad \important{\neg(\forall x) p(x) \;\Longleftrightarrow\; (\exists x) (\neg p(x))}
	\item $\neg( \exists x\in X: p(x))\quad\Longleftrightarrow\quad \forall x\in X: \neg p(x)$ 
    \qquad  of \qquad \important{\neg(\exists x) p(x) \;\Longleftrightarrow\; (\forall x) (\neg p(x))}
\end{enumerate}
\end{definition}
In woorden: 'niet er bestaat' is hetzelfde als 'voor alle geldt niet', \nl en ook: 'niet voor alle' is hetzelfde als 'er bestaat waarvoor niet'.

\section{Combinaties van kwantoren}

In één wiskundige formule kunnen meerdere kwantoren voorkomen, waarbij de volgorde soms erg belangrijk is.
\begin{exercise}
	Zij $P$ een verzameling potjes van allemaal onderling \textit{verschillende} formaten, $D$ de verzameling van de bij\-ho\-ren\-de deksels. Zij  $p(x,y)$ met $x \in D$ en $y \in P$ het predicaat `$x$ past op $y$'.

	Kies voor elk van volgende formules de overeenkomstige Nederlandse zin, en zeg of de bewering waar of onwaar is.

\renewcommand{\isA}{}
\renewcommand{\isB}{}
\renewcommand{\isC}{}
\renewcommand{\isD}{}
\renewcommand{\isE}{}
\renewcommand{\isF}{}
\renewcommand{\isG}{}
\renewcommand{\isH}{}
\renewcommand{\isI}{}


% dropdown-choice 
\newcommand{\myopts}{\hfill\wordChoice{
   \choice[\isA]{A: Op elk potje past een dekseltje}
   \choice[\isB]{B: Er is een potje waar elk dekseltje op past}
   \choice[\isC]{C: Er is een potje waarop geen dekseltje past}
   \choice[\isD]{D: Er is een dekseltje dat op geen enkel potje past}
   \choice[\isE]{E: Er is een dekseltje dat op elk potje past}
   \choice[\isF]{F: Er is een dekseltje dat op een potje past}
   \choice[\isG]{G: Elk dekseltje past op elk potje}
   \choice[\isH]{H: Elk dekseltje past op een potje}
   }
}

% maar in PDF zijn die lange opties niet handig/nodig
\pdfOnly{
\renewcommand{\myopts}{\hfill\wordChoice{\choice[\isA]{A}\choice[\isB]{B}\choice[\isC]{C}\choice[\isD]{D}\choice[\isE]{E}\choice[\isF]{F}\choice[\isG]{G}\choice[\isH]{H}}}
}


\begin{enumerate}%
		\item[A] Op elk potje past een dekseltje
		\item[B] Er is een potje waar elk dekseltje op past
		\item[C] Er is een potje waarop geen dekseltje past
		\item[D] Er is een dekseltje dat op geen enkel potje past
		\item[E] Er is een dekseltje dat op elk potje past
		\item[F] Er is een dekseltje dat op een potje past		
		\item[G] Elk dekseltje past op elk potje
		\item[H] Elk dekseltje past op een potje
\end{enumerate}%

		\begin{question} $\forall x \in D : \forall y \in P : p(x,y)$\renewcommand{\isG}{correct}\myopts\choiceFalse\end{question}	
		\begin{question} $\forall y \in P : \forall x \in D : p(x,y)$\renewcommand{\isG}{correct}\myopts\choiceFalse\end{question}
		\begin{question} $\exists x \in D : \exists y \in P : p(x,y)$\renewcommand{\isF}{correct}\myopts\choiceTrue\end{question}
		\begin{question} $\exists y \in P : \exists x \in D : p(x,y)$\renewcommand{\isF}{correct}\myopts\choiceTrue\end{question}
		\begin{question} $\exists x \in D : \forall y \in P : p(x,y)$\renewcommand{\isE}{correct}\myopts\choiceFalse\end{question}
		\begin{question} $\forall y \in P : \exists x \in D : p(x,y)$\renewcommand{\isA}{correct}\myopts\choiceTrue\end{question}
		\begin{question} $\forall x \in D : \exists y \in P : p(x,y)$\renewcommand{\isH}{correct}\myopts\choiceTrue\end{question}
		\begin{question} $\exists y \in P : \forall x \in D : p(x,y)$\renewcommand{\isB}{correct}\myopts\choiceFalse\end{question}
		\begin{question} $\exists a \in P : \forall b \in D : p(a,b)$\renewcommand{\isB}{correct}\myopts\choiceFalse\end{question}
		\begin{question} $\exists x \in P : \forall y \in D : p(y,x)$\renewcommand{\isB}{correct}\myopts\choiceFalse\end{question}

\end{exercise}

Deze oefening kan worden veralgemeend in volgende eigenschappen:
\begin{proposition}[Verwisselen van kwantoren]
Zij $p(x,y)$ een predicaat voor $x \in X$ en $y \in Y$, dan geldt:
\begin{enumerate}
	\item $\forall x \in X : \forall y \in Y : p(x,y)\quad\Longleftrightarrow\quad \forall y \in Y : \forall x \in X : p(x,y)$
	\item $\exists x \in X : \exists y \in Y : p(x,y)\quad\Longleftrightarrow\quad \exists y \in Y : \exists x \in X : p(x,y)$
	\item $\exists x \in X : \forall y \in Y : p(x,y)\quad\Longrightarrow\quad \forall y \in Y : \exists x \in X : p(x,y)$

	\item $ \forall y \in Y : \exists x \in X : p(x,y) \quad\xcancel{\vphantom{x^2_2}\Longrightarrow}\quad    \exists x \in X : \forall y \in Y : p(x,y)$		
\end{enumerate}
    Dit betekent:
    \begin{itemize}
    \item Gelijksoortige kwantoren kunnen altijd van plaats worden verwisseld, de bewering blijft waar.
    \item Een bestaanskwantor mag achter een alkwantor worden geschoven, de bewering blijft waar, maar wordt zwakker.
    \item Een alkwantor \textbf{mag niet} achter een bestaanskwantor worden geschoven, de bewering wordt dan mogelijk vals.
    \end{itemize}

\end{proposition}

\begin{remark}
	De ervaring leert dat in een wiskundige context regelmatig kwantoren verkeerdelijk worden omgewisseld, terwijl dat (hopelijk) nooit wordt gedaan in het gewone taalgebruik (denk aan de potjes en dekseltjes).   
    Mogelijk dringt de betekenis van formeel genoteerde uitspraken met kwantoren niet voldoende door, en ziet men eerder een hi\"eroglyfen-opschrift op een Egyptische tempel dan een uitspraak met een concrete betekenis.

    Het is erg belangrijk deze compact genoteerde uitspraken (en bij uitbreiding alle uitdrukkingen met wiskundige symbolen) te beschouwen als normale volzinnen. 
    %Het lezen en begrijpen van Nederlandse volzinnen is trouwens in veel gevallen aanzienlijk moeilijker en subtieler dan het lezen en begrijpen van --typisch duidelijk gestructureerde-- wiskundige formules. Maar, het vraagt oefening om er mee vertrouwd te worden.

\end{remark}


\end{document}