% Shortcuts voor limieten
% MERK OP: hier kan dus ook de notatie voor linker/rechterlimiet worden gekozen !!!
% Usage: \limx geeft lim voor x-> 0;  \limx[a^2]  geeft lim voor x-> a^2 en \limxi geeft lim voor x -> \infy \limxmi -> -\infty 

% Mmm, zonder de \ifblank lijkt het niet te werken in htlatex ...?
%\newcommand{\limx}[1][]{\lim_{x \rightarrow \ifblank{#1}{0}{#1}}}

\newcommand{\limxg}[1]{\underset {x \to  #1} \lim}    % g  voor Generic ...
\newcommand{\llimxg}[1]{\underset {x \underset < \to  #1} \lim} 
\newcommand{\rlimxg}[1]{\underset {x \underset > \to  #1} \lim}

\newcommand{\limx}{\limxg{0}}     % limieten naar 0
\newcommand{\llimx}{\llimxg{0}}
\newcommand{\rlimx}{\rlimxg{0}}

\newcommand{\limxi}{\limxg{+\infty}}  % I voor \Infty
\newcommand{\limxmi}{\limxg{-\infty}} % MI voor Min \Infty
\newcommand{\limxc}{\limxg{c}}     % is/was handig in de module over limieten ...
\newcommand{\llimxc}{\llimxg{c}}
\newcommand{\rlimxc}{\rlimxg{c}}

% to be moved to preamble ...?
\providecommand{\p}{} % default nothing ; potentially usefull for slides: redefine as \pause