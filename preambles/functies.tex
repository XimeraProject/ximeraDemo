%
% Voor functies en machine-voorstellingen van functies
%

\usetikzlibrary{graphs}
\usetikzlibrary{quotes}
\usetikzlibrary{arrows,positioning,decorations.pathreplacing}
%\usetikzlibrary{calc}

\usepackage{tikz-cd}

%
%
%  Ooit is geprobeerd om via makemathbox een box rond een functie te krijgen: werkt niet (goed) online (MathJax)
%\usepackage{mathtools} % makemathbox ... 
%
%% teken een box rond iets
%%\newcommand*{\mwbox}[1]{%
%%	\fbox{\makebox[{\ifdim 2em <\width\width\else 2em \fi}]{\vphantom{$x^2$}#1}}%
%%}
%
%\newcommand*{\mwbox}[1]{%
%	\fbox{\vphantom{$x^2$}#1}%
%}
%\newcommand{\fb}[1]{\mwbox{$\mathbf{#1}$}}
%\newcommand{\fb[1]}{\underline{\mathbf{#1}}}    % als de blokjes visueel te dominant zijn ?
%


% teken blokjes rond functies (in voorbeelden/oefeningen, om een visuele gelijkenis te creëren tussen de 'machine' en de 'formules')
% (voorlopig?)  via TiKZ (nadeel: ERG traag)
% minimal box, enkel bedoeld voor expliciete functie-notatie 
%  allicht niet te veel gebruiken ...?
%\tikzset{
%	f/.style={
%		minimum width=2em, 
%		minimum height= 2em,
%		font=\bfseries
%	},
%	fb/.style={f,rectangle,rounded corners,draw}
%}

\newcommand{\f}[1]{%
	\makebox[{\ifdim 1.5em <\width\width\else 1.5em \fi}]{\ensuremath{\vphantom{()x^2}\mathbf{#1}}}%
}%

%{\tikz[baseline=-1ex]{\node[f]{\ensuremath{\mathbf{#1}}};}}
%\newcommand{\fb}[1]{\tikz[baseline=-1ex]{\node[fb]{\ensuremath{\mathbf{#1}}};}}

\newcommand{\fb}[1]{\framebox{\f{#1}}}

\newcommand{\fc}{\fb{c}}    % shortcut, convenience ...
\newcommand{\fx}{\fb{x}}     % shortcut, convenience ...

% maak plus/maal/samenstellen even breed  (voor alignering bij definite, samen met de blokjes van heirboven)
\newcommand{\fp}{\f{+}} %{\ensuremath{\mathmakebox[1em]+}}       %plus
\newcommand{\fm}{\f{\cdot}}  %{\ensuremath{\mathmakebox[1em]\cdot}}   %maal
\newcommand{\fs}{\f{\circ}}  % {\ensuremath{\mathmakebox[1em]\circ}}   %samenstelling



% Styling voor functies-als-machines, op basis van de tikz 'graphs' library
\tikzset{
	fctie/.style={
		rectangle, 
		draw, 
		fill=blue!20, 
		node distance=1cm,     % wat is er hier precies nodig/nuttig ...?
		minimum width=1cm, 
		text centered, 
		rounded corners, 
		minimum height= 1cm,
		thick, 
		font=\bfseries
	}
}

% om dubieuze redenen wordt hier nu \tikzstyle gebruikt ipv \tikzset ...?
\tikzstyle{dom}   = [rounded rectangle,rounded rectangle west arc=none, draw, fill=blue!40, minimum width=1.5em, text centered, minimum height=2em, thick, font=\bfseries,]
\tikzstyle{codom} = [dom]   
%\tikzstyle{codom} = [dom,rounded rectangle east arc=none]   % als codom bv 'omgekeerd' wordt weergegeven tov dom


% vert is een misnoemer: dient voor plus en maal (dus: meerdere inputs; was ooit een 'vertikaal' blok)
% clone  is hetzelfde, maar dient voor meerdere outputs ...
\tikzstyle{vert}  = [fctie,fill=blue!5]
\tikzstyle{clone} = [fctie,fill=blue!5]

%
% Macro om grafieken te tekenen. 
% - eerste optioneel argument bv het domein van de functie (of elk extra argument aan \plot), 
% - eerst echt agrument: de te plotten functie
% - tweede echt argument : extra opties aan \begin{axis} (bv xmin/xmax etc)
% Voorbeeld: %\xmtoonfunctie[domain=1:2]{sqrt(x)}{xmin=-0.5,xmax=3.5,ymin=-0.5,ymax=2}
% TODO: nakijken of we x- en y-schaal gelijk willen ... (nu niet het geval, kan natuurlijk worden opgegeven in arg2)
 
\newcommand{\xmtoonfunctie}[3][]{
\begin{image}[0.5\textwidth]               
  \tikz[scale=1, baseline=(current bounding box.north)]{
  \begin{axis}[grid = major, xlabel = $x$, ylabel = $y$, restrict y to domain=-20:20, #3]

  \IfNoValueTF{#1}
	{\plot[very thick, blue] plot[samples = 500, smooth] expression{#2};} % geen domein gegeven: domein is [xmin, xmax]
	{\plot[very thick, blue, #1] plot[samples = 500, smooth] expression{#2};} % wel domein gegeven: gebruiken
\end{axis}
}
\end{image}
}
