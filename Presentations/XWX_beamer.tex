%Latex->PDF
%
% TRUCJE (afhankelijk van de bestandsnaam, krijg je de handout of de presentatie ...
% (!! GEBRUIK LINUX HARD LINKS OM HETZELFDE BESTAND TWEE NAMEN TE GEVEN !!)
% See https://tex.stackexchange.com/questions/197330/how-can-i-check-if-the-filename-of-a-latex-document-contains-a-string
%
\RequirePackage{substr}   % RequirePackage omdat \usepackage niet werkt voor \begin{document} !

\begingroup\escapechar=-1
\xdef\handoutstring{\string\handout}   % dirty hack, omdat anders stringcompare niet werkt
\endgroup

\IfSubStringInString{\handoutstring}{\jobname}
{
\documentclass[aspectratio=169,handout]{beamer}
\wlog{We genereren de HANDOUT versie voor \jobname.}
}{
\documentclass[aspectratio=169]{beamer}
\wlog{We genereren de PRESENTATIE versie voor \jobname.}
}
% EndofTrucje


\usepackage[dutch]{babel}
\usepackage{amssymb,amsmath,amsthm}

\usepackage{cancel}
\usepackage{relsize}

\usepackage{listings}
\usepackage{xcolor}
%\usepackage{verbatim}

\usepackage{qrcode}

\setbeamertemplate{headline}{\insertframenumber/\inserttotalframenumber}
\newcommand{\ds}{\displaystyle}
\newcommand{\N}{\ensuremath{\mathbb{N}}}
\newcommand{\Z}{\ensuremath{\mathbb{Z}}}
\newcommand{\Nnul}{\ensuremath{\mathbb{N}_0}}
\newcommand{\Q}{\ensuremath{\mathbb{Q}}}
\newcommand{\R}{\ensuremath{\mathbb{R}}}
\newcommand{\perdef}{\overset{\mathrm{def}}{=}}
\newcommand{\Bgsin}{\mathrm{Bgsin}\,}
\newcommand{\Bgcos}{\mathrm{Bgcos}\,}
\newcommand{\Bgtan}{\mathrm{Bgtan}\,}

\DeclareMathOperator{\dom}{dom}     % domein
\DeclareMathOperator{\codom}{codom} % codomein
\DeclareMathOperator{\bld}{bld}     % beeld
\DeclareMathOperator{\graf}{graf}   % grafiek
\DeclareMathOperator{\rico}{rico}   % richtingcoëfficient
\DeclareMathOperator{\co}{co}       % coordinaat

\newtheorem*{eigenschap}{Eigenschap}

\newcommand{\p}{\pause}
\newcommand{\important}[1]{\ensuremath{\colorbox{lightgray}{$#1$}}}

\providecommand{\blue}[1]{{\color{blue}#1}}  
\providecommand{\red}[1]{{\color{red}#1}}  

\usepackage{pgf,tikz,pgfplots}
%\pgfplotsset{compat=1.15}
\usepackage{mathrsfs}
\usepackage{dsfont}
\usepackage{relsize}   % voor \mathlarger
\usetikzlibrary{arrows}
 \usepgfplotslibrary{fillbetween}
 \usetikzlibrary{positioning, fit, calc} 
 \usetikzlibrary{intersections}
 \usetikzlibrary{shapes}
%\pagestyle{empty}



\usepackage{rawfonts}
%\input{prepictex}
%\input{pictex}
%\input{postpictex}

\graphicspath{
	{../../}
	{../}
	{./}
  	{../../pictures/}
   	{../pictures/}
   	{./pictures/}
}

\title{Ximera Deployment(s)}
% \date{\qrcode{https://github.com/XimeraProject}}
\author{Wim Obbels}
\date{May 12 2025}
%\usetheme{CambridgeUS}
\usetheme{Boadilla}
%\usecolortheme{seahorse}

\begin{document}

\tikzset{block/.style={draw, thick, text width=3cm , minimum height=2cm, % align=center,
     rectangle split, 
     rectangle split ignore empty parts, 
     rectangle split parts=3,
     rectangle split part align={center, left},
     rounded corners=0.2cm,
     draw},   
     line/.style={-latex}     
}  

\begin{frame}
  \titlepage
\end{frame}


\begin{frame}{Ximera: basic concepts}
  \begin{itemize}[<+->]
    \item "Dwarf on the shoulders of giants" (Bernard of Chartres, 12th century; I. Newton)
    \begin{itemize}[<+->]
      \item \LaTeX
      \item Tex4ht
      \item git, docker, VS code, ....
    \end{itemize}
    \item "There's more than one way to do it" (Perl)
  \end{itemize}
\end{frame}

\begin{frame}{Ximera: needed/useful competences}
  \begin{itemize}[<+->]
    \item Some mathematics
    \item Some didactical/educational knowledge
    \item \LaTeX
    \item git  (basic notions)
    \item (optional, but strongly suggested) VS Code   (editing/building)
    \item (optional, for local development) docker
  \end{itemize}
\end{frame}

\begin{frame}{Ximera: needed/useful software}
  \begin{itemize}[<+->]
    \item A browser
    \begin{itemize}[<+->]
      \item ... and nothing else !
    \end{itemize}
    \item (optional) For local development:
    \begin{itemize}[<+->]
      \item git
      \item VS Code
      \item docker
    \end{itemize}
    \item NOT NEEDED: a local \TeX\  installation
  \end{itemize}
\end{frame}


\begin{frame}[t]{Build Architecture ('codespaces DEV')}

  \begin{center}
    \resizebox{0.99\linewidth}{!}{
    \begin{tikzpicture}  
    
    \node[block, fill=white,text width=2cm] (pcenduser) {
              Student
              \nodepart{three}
              - browser
    };  

    \p
    \node[block, fill=green,below=3cm of pcenduser, text width=2cm ] (pcauthor) {
        Author
        \nodepart{three}
        - browser\\
        \phantom{- VS Code}\\
        \phantom{- VS Code}\\
        \phantom{- VS Code}\\
        \phantom{- VS Code}\\
        \phantom{- VS Code}\\
    };  

    \p
    \node[block, fill=orange!50!white,right=9.5cm of pcenduser,text width=8.5cm] (github) {
        Github Website (Repo for Code and Docker Images)
        \nodepart{two}
        - (optional) Clone ximeraFirstSteps (or other repo)\\
        - (re-)Start a Codespace\\
        \nodepart{three} 
        \url{https://github.com}\\
      };        

    \draw[dashed,line,->] (pcauthor.90) -- node[pos=0.2,below right] {} (github.200);

    \p
    \node[block,right=2cm of pcauthor, fill= green,text width=5cm] (ximeraserver) {
        DEV Ximera Server 
        \nodepart{two} 
        Runs image 'ximeraserver'
        \nodepart{three} 
        \url{https://<codespace-name>-2080.app.github.dev}\\
    };       

    \node[block, fill=green,right=2cm of ximeraserver,text width=8cm] (ximeralatex) {
    Github Codespace
        \nodepart{two} 
        Runs image 'ximeralatex'\\%\hline\\
        - edit \LaTeX code (in VSCode in browser)\\
        - Generate GPG Key (from 'Extra' menu)\\
        - Push 'SERVE' button \\
        - (optional) \textbf{git commit/push}
        \nodepart{three}
            \url{https://<codespace-name>.github.dev}
        };  

     

    \node[draw, fill=gray, fill opacity=0.1,inner xsep=3mm,inner ysep=3mm,
        fit=(ximeraserver)(ximeralatex),
        label={-90:Codespace (VS Code runs in browser, with docker on Github infrastructure)}] (codespace) {};  

    \draw[line,<-] (codespace) -- node[left] {Start Codespace} (github);  

    \draw[dashed,line,->] (pcauthor.270) -- node[pos=0.2,below right] {} (ximeralatex.200);
    \p

\draw[line,->,red] (ximeralatex) -- node[above] {SERVE} (ximeraserver);  
\draw[dashed,line,->,red] (pcauthor.0) -- node[pos=0.2,below right] {} (ximeraserver.200);
\p
\draw[line,->] (ximeralatex) -- node[right] {push/pull} (github);  


\end{tikzpicture}
  }
\end{center}
\end{frame}



\begin{frame}[t]{Build Architecture ('codespaces PROD-MANUAL')}

  \begin{center}
    \resizebox{0.99\linewidth}{!}{
    \begin{tikzpicture}  
    
    \node[block, fill=green,text width=2cm] (pcenduser) {
              Student
              \nodepart{three}
              - browser
    };  

    \node[block, fill=green,below=3cm of pcenduser, text width=2cm ] (pcauthor) {
        Author
        \nodepart{three}
        - browser\\
        \phantom{- VS Code}\\
        \phantom{- VS Code}\\
        \phantom{- VS Code}\\
        \phantom{- VS Code}\\
        \phantom{- VS Code}\\
    };  

    \node[block, fill=orange!50!white,right=9.5cm of pcenduser,text width=8.5cm] (github) {
        Github Website (Repo for Code and Docker Images)
        \nodepart{two}
        - (optional) Clone ximeraFirstSteps (or other repo)\\
        - (re-)Start a Codespace\\
        \nodepart{three} 
        \url{https://github.com}\\
      };        


    \node[block,right=1cm of pcenduser, fill=orange!50!white,text width=7cm] (ximeraprodserver) {
        PRODUCTION Ximera Server 
        \nodepart{two} 
        Runs image 'ximeraserver'
        \nodepart{three} 
        \url{https://ximera.osu.edu}\\
        \url{https://set.kuleuven.be/voorkennis}\\
    };       

    \node[block,right=2cm of pcauthor, fill= green,text width=5cm] (ximeraserver) {
        DEV Ximera Server 
        \nodepart{two} 
        Runs image 'ximeraserver'
        \nodepart{three} 
        \url{https://<codespace-name>-2080.app.github.dev}\\
    };       

    \node[block, fill=green,right=2cm of ximeraserver,text width=8cm] (ximeralatex) {
    Github Codespace
        \nodepart{two} 
        Runs image 'ximeralatex'\\%\hline\\
        - edit \LaTeX code (in VSCode in browser)\\
        - Generate GPG Key (from 'Extra' menu)\\
        - Push 'SERVE' button \\
        - (optional) \textbf{git commit/push}
        \nodepart{three}
            \url{https://<codespace-name>}
        };

\node[draw, fill=gray, fill opacity=0.1,inner xsep=3mm,inner ysep=3mm,
        fit=(ximeraserver)(ximeralatex),
        label={-90:Codespace}] (codespace) {};  


\draw[dashed,line,->] (pcauthor.90) -- node[pos=0.2,below right] {} (github.200);
\draw[dashed,line,->] (pcauthor.0) -- node[pos=0.2,below right] {} (ximeraserver.200);
\draw[dashed,line,->] (pcauthor.270) -- node[pos=0.2,below right] {} (ximeralatex.200);

\draw[line,->] (ximeralatex) -- node[above] {SERVE} (ximeraserver);  
\draw[line,->,red] (ximeralatex.180) -- node[right] {SERVE} (ximeraprodserver);  
\draw[dashed,line,->,red] (pcenduser) -- node[right] {} (ximeraprodserver);  

\draw[line,->] (ximeralatex) -- node[right] {push/pull} (github);  
% \draw[line,<-] (ximeralatex.110) -- node[left] {Start Codespace} (github.240);  
    \draw[line,<-] (codespace) -- node[left] {Start Codespace} (github);  

\end{tikzpicture}
  }
\end{center}
\end{frame}



\begin{frame}[t]{Build Architecture ('codespaces PROD-ACTION')}

  \begin{center}
    \resizebox{0.99\linewidth}{!}{
    \begin{tikzpicture}  
    
    \node[block, fill=green,text width=2cm] (pcenduser) {
              Student
              \nodepart{three}
              - browser
    };  

    \node[block, fill=green,below=3cm of pcenduser, text width=2cm ] (pcauthor) {
        Author
        \nodepart{three}
        - browser\\
        \phantom{- VS Code}\\
        \phantom{- VS Code}\\
        \phantom{- VS Code}\\
        \phantom{- VS Code}\\
        \phantom{- VS Code}\\
    };  

    \node[block, fill=orange!50!white,right=9.5cm of pcenduser,text width=8.5cm] (github) {
        Github Website (Repo for Code and Docker Images)
        \nodepart{two}
        - (optional) Clone ximeraFirstSteps (or other repo)\\
        - (re-)Start a Codespace\\
        \nodepart{three} 
        \url{https://github.com}\\
      };        

    \node[block,right=1cm of pcenduser, fill=orange!50!white,text width=7cm] (ximeraprodserver) {
        PRODUCTION Ximera Server 
        \nodepart{two} 
        Runs image 'ximeraserver'
        \nodepart{three} 
        \url{https://ximera.osu.edu}\\
        \url{https://set.kuleuven.be/voorkennis}\\
    };       

    \node[block,right=2cm of pcauthor, fill= green,text width=5cm] (ximeraserver) {
        DEV Ximera Server 
        \nodepart{two} 
        Runs image 'ximeraserver'
        \nodepart{three} 
        \url{https://<codespace-name>-2080.app.github.dev}\\
    };       

    \node[block, fill=green,right=2cm of ximeraserver,text width=8cm] (ximeralatex) {
    Github Codespace
        \nodepart{two} 
        Runs image 'ximeralatex'\\%\hline\\
        - edit \LaTeX code (in VSCode in browser)\\
        - Generate GPG Key (from 'Extra' menu)\\
        - Push 'SERVE' button \\
        - (optional) \textbf{git commit/push}
        \nodepart{three}
            \url{https://<codespace-name>}
        };

\node[draw, fill=gray, fill opacity=0.1,inner xsep=3mm,inner ysep=3mm,
        fit=(ximeraserver)(ximeralatex),
        label={-90:Codespace}] (codespace) {};  


\draw[dashed,line,->] (pcauthor.90) -- node[pos=0.2,below right] {} (github.200);
\draw[dashed,line,->] (pcauthor.0) -- node[pos=0.2,below right] {} (ximeraserver.200);
\draw[dashed,line,->] (pcauthor.270) -- node[pos=0.2,below right] {} (ximeralatex.200);

\draw[line,->] (ximeralatex) -- node[above] {SERVE} (ximeraserver);  
\draw[line,->,red] (github) -- node[above] {SERVE} node[below]{action} (ximeraprodserver);  
\draw[dashed,line,->,red] (pcenduser) -- node[right] {} (ximeraprodserver);  

\draw[line,->] (ximeralatex) -- node[right] {push/pull} (github);  
%\draw[line,<-] (ximeralatex.110) -- node[left] {Start Codespace} (github.240);  
    \draw[line,<-] (codespace) -- node[left] {Start Codespace} (github);  

\end{tikzpicture}
  }
\end{center}
\end{frame}

\begin{frame}[t]{Build Architecture ('LOCAL DOCKER')}

  \begin{center}
    \resizebox{0.99\linewidth}{!}{
    \begin{tikzpicture}  
    
    \node[block, fill=green,text width=2cm] (pcenduser) {
              Student
              \nodepart{three}
              - browser
    };  

    \node[block, fill=green,below=3cm of pcenduser, text width=2cm ] (pcauthor) {
        Author
        \nodepart{three}
        - browser\\
        \phantom{- VS Code}\\
        - git\\
        - VS Code\\
        - docker\\
    };  

    \p

    \node[block, fill=orange!50!white,right=9.5cm of pcenduser,text width=8.5cm] (github) {
        Github Website (Repo for Code and Docker Images)
        \nodepart{two}
        \phantom{- (optional) Clone ximeraFirstSteps (or other repo)}\\
        \phantom{- (re-)Start a Codespace}\\
        \nodepart{three} 
        \url{https://github.com}\\
      };        

    \draw[dashed,line,->] (pcauthor.90) -- node[pos=0.5,above left] {clone} (github.200);
    \p

    
    \node[block,right=2cm of pcauthor, fill= green,text width=5cm] (ximeraserver) {
        LOCAL Ximera Server 
        \nodepart{two} 
        Runs image 'ximeraserver'
        \nodepart{three} 
        \url{https://localhost:2000}\\
    };       

    \node[block, fill=green,right=2cm of ximeraserver,text width=8cm] (ximeralatex) {
    VSCode 
        \nodepart{two} 
        Starts image 'ximeralatex'\\%\hline\\
        - edit \LaTeX code (in VSCode in browser)\\
        - Generate GPG Key (from 'Extra' menu)\\
        - Push 'SERVE' button \\
        - (optional) \textbf{git commit/push}
        \nodepart{three}
            \phantom{\url{https://<codespace-name>}}
        };

\node[draw, fill=gray, fill opacity=0.1,inner xsep=0mm,inner ysep=3mm,
        fit=(ximeraserver)(ximeralatex)(pcauthor),
        label={-90:VScode + docker}] (codespace) {};  

\p
\draw[line,->,red] (ximeralatex.180) -- node[above] {SERVE} (ximeraserver);  
\p
\node[block,right=1cm of pcenduser, fill=orange!50!white,text width=7cm] (ximeraprodserver) {
        PRODUCTION Ximera Server 
        \nodepart{two} 
        Runs image 'ximeraserver'
        \nodepart{three} 
        \url{https://ximera.osu.edu}\\
        \url{https://set.kuleuven.be/voorkennis}\\
    };       

    \p
    \draw[line,->,red] (ximeralatex.180) -- node[pos=0.6,above right] {SERVE}  (ximeraprodserver);  
    \p
    \draw[line,->,red] (github) -- node[above] {SERVE} node[below]{action} (ximeraprodserver);  
\draw[line,->] (ximeralatex) -- node[right] {push/pull} (github);  
\draw[dashed,line,->,red] (pcenduser) -- node[right] {} (ximeraprodserver);  



% \draw[dashed,line,->] (pcauthor.90) -- node[pos=0.2,below right] {} (github.200);
% \draw[dashed,line,->] (pcauthor.0) -- node[pos=0.2,below right] {} (ximeraserver.200);
% \draw[dashed,line,->] (pcauthor.270) -- node[pos=0.2,below right] {} (ximeralatex.200);


%\draw[line,<-] (ximeralatex.110) -- node[left] {Start Codespace} (github.240);  
%    \draw[line,<-] (codespace) -- node[left] {Start Codespace} (github);  

\end{tikzpicture}
  }
\end{center}
\end{frame}


\end{document}