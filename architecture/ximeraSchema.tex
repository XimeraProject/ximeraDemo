\documentclass{ximera}
\input{../preamble.tex}

\addPrintStyle{..}

\begin{document}
    \author{Wim Obbels}
	\xmtitle{Ximera Schematisch Overzicht}{}
    \label{xim:ximeraSchema}

    
\subsection{Inleiding}

Ximera is een open source \LaTeX\  module, ontwikkeld aan Ohio State University, die twee nieuwe \textit{documentclasses} \verb|ximera| en \verb|xourse| definieert.
Hierbij komen enkele nieuwe \textit{macro's} en nieuwe \textit{environments} beschikbaar met extra functionaliteit om van .tex bestanden niet alleen PDF's te genereren, maar ook interactieve HTML webpagina's. Voor de HTML-versie wordt gebruik gemaakt van \href{https://tug.org/tex4ht/}{TeX4ht}, vooral de interactiviteit, opbouw en integratie met een eigen webserver is specifiek Ximera.

Aan de KU Leuven is de Ximera-broncode aangepast en verder ontwikkeld, en wordt het systeem gebruikt voor voorbereidingstrajecten wiskunde, fysica en chemie.

Ximera bevat dus enerzijds een TeX module met de documentclasses \verb|ximera| en \verb|xourse|, maar anderzijds ook een buildomgeving ('xake', naar analogie met 'make' voor bv C-programma's) en een webserver waarop de cursussen automatisch worden gepubliceerd, en waar een LTI-koppeling met bv. Toledo mogelijk is.

\subsection{Ximera webserver}

Voor eindgebruikers zijn er twee Ximera-webservers, bereikbaar via paden onder

\url{https://set.kuleuven.be/voorkennis} (default pad) en

\url{https://set.kuleuven.be/ximera-wis} (voor monitoraat).

De volledige urls van beschikbare cursussen worden ofwel aangeboden via de KU Leuven website (bv via de \href{https://wet.kuleuven.be/schoolverlater/basiswiskunde}{zomercursuspagina's van Wetenschappen}), en dan is de toegang publiek en anoniem. 
Ofwel worden de links via Toledo(-cursussen of -communities) aangeboden, al dan niet met een LTI koppeling. Bij gebruik van een LTI-koppeling zijn gebruikers automatisch ook in Ximera geauthenticeerd, en worden gedetailleerde statistieken bijgehouden.


Beide Ximera-servers draaien in een docker-container op de Cloud Infrastructuur van de KU Leuven, beheerd via

\url{https://elsschot.icts.kuleuven.be}

De docker-images zijn te vinden via (access: via ICTS Cloud)

\url{ https://repo.icts.kuleuven.be/ui/repos/tree/General/icts-p-icts-registry-ext-docker-local/set/dsb}


De broncode van de Ximera-webserver is te vinden op 

\url{https://gitlab.kuleuven.be/monitoraat-wet/Ximera-server}

Meer info zie \hyperref[sec:ximera-webserver]{verder}.

\subsection{Ximera TeX module voor auteurs}

De inhoud van de Ximera-cursussen is in TeX geschreven, en moet in een git-repository worden bewaard. Ximera publiceert dan een bepaalde commit van dergelijk repo op een Ximera-webserver. 
Dat process verloopt aan de KU Leuven volledig geautomatiseerd via gitlab pipelines: bij elke commit wordt automatisch de recentste versie gepubliceerd op een test-versie van de website, en via een kleine manuele interventie kan die al dan niet op de publieke website worden gezet.

Auteurs van Ximera-inhoud kunnen triviale wijzigingen aan de TeX-code rechtstreeks via de gitlab-website aanbrengen. Bij commit wordt alles gecompileerd, en eventuele compilatie-fouten zijn raadpleegbaar via de Pipeline-pagina's van gitlab.

Voor uitgebreide wijzigingen is deze werkwijze niet effectief, en is het nodig het git-repo met de TeX code te clonen, en lokaal de wijzigingen aan te brengen en te testen. 

Daarvoor moet ofwel een lokale TeX-installatie beschikbaar zijn, met bovendien de Ximera TeX-module zoals te vinden op \url{https://gitlab.kuleuven.be/monitoraat-wet/ximeraLatex}. Alternatief kan worden gewerkt met docker en de docker-image zoals gegeneerd via \url{https://gitlab.kuleuven.be/monitoraat-wet/xake}.

In beide gevallen moeten de wijzigingen worden gecommit en gepushed, en zorgt de gitlab-pipeline voor publicatie.

\subsection{Ximera build omgeving via gitlab}

Elk repository met Ximera TeX-code bevat ook een .gitlab-ci.yml bestand dat definieert hoe en waar de code moet worden gepubliceerd.
Dat gebeurt typisch via een script build.sh, dat zelf de Ximera-buildtool xake oproept en

\begin{itemize}
    \setlength\itemsep{0em}
    \item .html bestanden genereert  (via \texttt{xake bake})
    \item PDF's genereert (via\texttt{xake bakePdf})
    \item deze publiceert naar een test-versie op een opgegeven Ximera server (via \texttt{xake frost} en \texttt{xake serve})
    \item een manuele actie voorziet om de website ook effectief te publiceren
\end{itemize}

De xake-executable zit in een container, waarvan de code te vinden is op  \url{https://gitlab.kuleuven.be/monitoraat-wet/xake}, en waarvan het image gepubliceerd is op \url{ https://repo.icts.kuleuven.be}.

De pipeline wordt uitgevoerd door gitlab 'runners' die centraal op de gitlab-server beschikbaar zijn, en automatisch de juiste docker-image pullen.

De authenticatie en authorisatie voor publiceren zit volledig in GITLAB: daar wordt beslist wie lees en/of schrijfrechten heeft, en wie publiceerrechten heeft (voorlopig direct gekoppeld aan schrijfrechten).

\begin{xmuitweiding}[Mogelijke alternatieven:]\nl
    
\begin{enumerate}
    \item de Auteur genereert en publiceert de code zelf (via 'docker xake')
    \\ Nadeel: authenticatie en authorisatie nodig om op de Ximera Server wijzigingen aan te brengen; traceerbaarheid;  reproduceerbaarheid, ...
    \item de Auteur start zelf een 'GITLAB runner'  (via 'docker gitlab-runner')
    \\ Nadeel: authenticatie en autorisatie nodig om GITLAB runners te starten; mogelijk stabiliteitsprobleem  (met het docker-image voor een GITLAB runner op een Windows-PC; de linux versie is bijzonder stabiel ;-) ).
    
\end{enumerate}
\end{xmuitweiding}


\subsection{Overzicht}


%\usetikzlibrary{positioning, fit, calc}   

\tikzset{block/.style={draw, thick, text width=3cm , minimum height=2cm, % align=center,
     rectangle split, 
     rectangle split ignore empty parts, 
     rectangle split parts=3,
     rectangle split part align={center, left},
     rounded corners=0.2cm,
     draw},   
line/.style={-latex}     
}  

  
%\pdfOnly{\begin{landscape} }
  
\begin{image}[\textwidth]
\begin{tikzpicture}  

\node[block, fill=green] (pceind) {
          PC Eindgebruiker
          \nodepart{two}
          - browser
         };  


\node[block, fill=orange!50!white, below=of pceind ] (pc1) {
    PC Auteur 1
    \nodepart{two}
    - browser \\          
    - git client\\ 
    - docker client
    \nodepart{three} Repositories:\\
    - clone cursus1.tex    
    }; 

\node[block, fill=orange!50!white, below=0.2cm of pc1 ] (pc2) {
    PC Auteur 2
    \nodepart{two}
    - browser \\          
    (via Gitlab IDE)
}; 


\node[block, fill=orange!50!white,below=0.2cm of pc2  ] (pc3) {
    PC Auteur 3
    \nodepart{two}
     - browser \\
     - git client\\ 
     - LaTeX client
    \nodepart{three} Repositories:\\
    - clone cursus1.tex \\
    - clone ximera.cls
    };  

     
\node[block,right=of pceind, fill= green,text width=11.5cm] (ximeraserver) {
  Ximera Server (publieke website, via de ICTS Cloud en docker)
  \nodepart{three} \url{set.kuleuven.be/voorkennis/zomercursus}
  };  
%
\node[block, below right=2cm of pceind, fill=orange,text width=5cm ] (gitrunner) {  % geknoei
    GITLAB runner
    \nodepart{two}
    Genereert PDF en HTML \\
    naar Ximera Server (xake)
}; 

\node[block, right=of gitrunner, fill=orange,text width=5cm] (elsschot) {
    Cloud Management (docker)
    \nodepart{three} \url{elsschot.icts.kuleuven.be}
};  


\node[block, fill=orange!50!white,below =of gitrunner,text width=5cm     ] (gitrepo) {
  GITLAB Repository server
  \nodepart{two}
    met git-repositories ((broncode):\\
    - repo cursus1.tex\\
    - repo cursus2.tex\\
    - repo ximera.cls\\
    - repo ximeraserver.js\\
    - repo xake.go\\
    - repo gitlabrunner.docker\\
   updates publiceren via GITLAB runner\\
   webpagina's beheer/diagnose\\ 

    \nodepart{three}
       \url{gitlab.kuleuven.be}
    };  


\node[block, right=of gitrepo, fill=orange,text width=5cm ] (dockerrepo) {
    Docker Repository Server \\
    ( Artifactory / FROG )
    \nodepart{two}met docker images \\
    - ximeraserver\\
    - xake\\
%    - gitlabrunner
    \nodepart{three} \url{repo.icts.kuleuven.be}
};  


%\node[block, fill=orange , below=of pceind ] (pcbeheer) {
    \node[block, fill=orange , below right=of ximeraserver] (pcbeheer) {
        PC Beheerder
        \nodepart{two}
        - browser \\          
        - git client\\
        - docker client
        \nodepart{three} Repositories:\\
        - clone ximeraserver \\
        - clone gitlabrunner
    };  
    


  
% Kader ICTS 

\node[draw, fill=gray, fill opacity=0.1,inner xsep=5mm,inner ysep=6mm,
      fit=(ximeraserver)(dockerrepo)(elsschot)(gitrepo)(gitrunner),
      label={90:ICTS Cloud / SET-IT}] (icts) {};   

% Connecties ...
\draw[line,->] (pceind)-- (ximeraserver);  
          
\draw[line,<->] (pc2) -- (pc2-|gitrepo.west);  
%\draw[line,<->] (pc1) -- (pc1-|gitrepo.west); 
\draw[line,<->] (pc3) -- (pc3-|gitrepo.west);  
\draw[line, ->,very thick,orange!50!white] (pc1.south east) -- node[black,above]{update} (gitrepo.north west);
%\draw[line, ->] (pc2.north east)-- (ximeraserver.west);

\draw[line,<-,very thick, orange]  (ximeraserver.south-|elsschot) -- node[black,right]{beheer} (elsschot); 
\draw[line,<->] (elsschot) -- (dockerrepo); 

\draw[line,->] (pcbeheer) |- (ximeraserver.east);  
\draw[line,->] (pcbeheer) |- (dockerrepo.east);  
\draw[line,->] (pcbeheer) -- (elsschot);  
\draw[line,->] (pcbeheer.210) -- (gitrepo.north east);  

\draw[line,->,very thick,orange!50!white] (gitrunner)                -- node[black,right,pos=0.5]{update} (gitrunner.north |- ximeraserver.south);  
\draw[line,->,very thick,orange!50!white] (gitrunner.south-|gitrepo) -- node[black,left]{pull} 
(gitrepo); 
\draw[line,-> ] (gitrunner.south-|dockerrepo) -- (dockerrepo);  

% Legende
\node[block, fill=orange!50!white,below =4cm of pcbeheer  ] (la) {Voor Auteurs};
\node[block, fill=orange,         below=0.1cm of la         ] (lb) {Voor Beheerders};
\node[block, fill=green ,         above=0.1cm of la         ] (lb) {Voor Eindgebruikers};

\end{tikzpicture}  
\end{image}





%\pdfOnly{\end{landscape} }


\end{document}
