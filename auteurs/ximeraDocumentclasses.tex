%%
%% Generated by gpt_translate from auteurs/ximeraDocumentclasses.tex, on 2024-07-14 18:59:07 using model gpt-3.5-turbo-16k
%%

% GPT CHUNK%
\documentclass{ximera}
\input{../preamble.tex}
\addPrintStyle{..}

\begin{document}
	\author{Wim Obbels}
	\xmtitle{Document classes 'ximera' and 'xourse'}{}
	\label{xim:ximeraDocumentclasses}

Ximera defines two new document classes, which are technically extensions of 'article'.

\subsection{Document class 'ximera'}

Ximera structures learning units into 'activities' (\verb|activity|), which are standalone \LaTeX~files with \verb|\documentclass{ximera}|. Each 'activity' document becomes one web page and is meant to achieve one or more well-defined learning goals (\verb|outcomes| in Ximeran).

Examples can be: \verb|fractions.tex| deals with the definition and basic properties of fractions (including some examples and exercises, because otherwise you can't learn it), while \verb|fractions_exercises.tex| contains extra exercises, \verb|fractions_challengingexercises.tex| for example some non-trivial problems, and \verb|fractions_polynomials.tex| gives extra theory. By convention, the exercises are in a subfolder \verb|exercises|.

Each \verb|\documentclass{ximera}| forms such an 'activity': a clearly defined and in principle standalone piece of teaching material.

The terms 'activity' and 'ximera' are synonymous.

{\footnotesize
\begin{verbatim}
\documentclass{ximera}          % this is an 'ximera' or 'activity'
\input{../preamble.tex}         % extra macros (from parent folder)
\addPrintStyle{..}              % extra settings only for PDF

\begin{document}
    \xmtitle{Introduction to fractions}{}     % set and print title

\begin{definition}
	A \textbf{fraction} is an expression of the form $\frac{a}{b}$, with $a,b\in\Z$.
\end{definition}
\end{document}
\end{verbatim}
}

\subsection{Document class 'xourse'}

A \verb|\documentclass{xourse}| is a \textit{course}, and consists of a list of 'activities'. A \verb|xourse| is essentially a list of 'includes' of \verb|ximera|'s, possibly including a specific introduction or preface. Therefore, a course is a particularly simple tex document, and it is consequently also very easy to compile new xourses based on existing activities.

{\footnotesize
\begin{verbatim}
\documentclass{xourse}       % this is a 'xourse'
\def\isXourse{true}          % necessary for technical reasons ...
\input{preamble.tex}         % defines various extra macros and environments
\iftikzexport\else           % for technical reasons only if NOT tikzexport ...
\usepackage{printstyle}       % defines various settings only for the PDF version
\fi

\begin{document}
    \xmtitle{My first Ximera course}{}        % set and print title (from preamble.tex)

    \part{Introduction}                             % standard TeX command
    \activitychapter{math/introduction.tex}      % include introduction as 'chapter'
    \activitychapter{math/fractions.tex}
    \practicesection{math/exercises/set1.tex} % include exercises as (sub-)section
    \activitychapter{math/powers.tex}

\end{document}
\end{verbatim}
}
\end{document}