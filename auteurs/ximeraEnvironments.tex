%%
%% Generated by gpt_translate from auteurs/ximeraEnvironments.tex, on 2024-07-14 18:59:35 using model gpt-3.5-turbo-16k
%%

% GPT CHUNK%
\documentclass{ximera}
\input{../preamble.tex}
\addPrintStyle{..}

\begin{document}
    \author{Wim Obbels}
    \xmtitle{Environments}{}
    \label{xim:ximeraEnvironments}

Ximera provides several environments with appropriate layout or functionality in both PDF and online:

\begin{enumerate}
    \setlength\itemsep{0em}
    \item theorem-like environments 
    \verb|definition|, \verb|proposition|, \verb|theorem|, 
    \verb|example|, \verb|remark|, \verb|warning|, 
    \\
    and also 
    \verb|notation|, \verb|axiom|, 
    \verb|proof|, \verb|explanation|, 
    \verb|observation|, \verb|algorithm|, \verb|claim|, \verb|conclusion|, \verb|condition|, \verb|conjecture|, \verb|corollary|, \verb|criterion|, \verb|fact|, \verb|lemma|, \verb|formula|, \verb|idea|, \verb|model|, \verb|paradox|, \verb|prodecure|, \verb|summary|, \verb|template|:

    These environments determine, as is standard in \LaTeX, the header, layout, and numbering of various parts of the text. Ximera also provides an HTML styling for these environments.

    By default, 'theorie' such as definition, proposition, etc. is greenish, and remark, warning, etc. are yellowish. The examples and exercises (see below) are bluish.

    Example, where an optional title can be given as usual in \LaTeX:

    \textbackslash \verb|begin{definition}[title] CONTENT |\textbackslash \verb|end{definition}|
    \begin{definition}[title]
        CONTENT
    \end{definition}

    \begin{proposition}[title]
        CONTENT
    \end{proposition}

    \begin{remark}[title]
        CONTENT
    \end{remark}

    \begin{example}[title]
        CONTENT
    \end{example}

    \begin{exercise}[title]
        CONTENT
    \end{exercise}

    \item problem-like environments \verb|problem|, \verb|exercise|, \verb|exploration|, \verb|question|:

    These problem environments function as the above, but they can also contain commands and other environments that generate online interactive components on which \textit{answers} can be entered. These answers are sent to the Ximera server, allowing for the creation of reports and analyses.
    % This last functionality can be integrated with LTI, e.g. with Toledo.

    In the PDF, either the correct answers are shown directly, or drop-down menus or input points are shown (depending on options that are enabled or disabled during compilation).

    Use \verb|example| for examples, and \verb|exercise| for exercises.
    The environments \verb|problem| and \verb|exploration| are currently not used or rarely used.

    The \verb|question| environment is modified to have a minimal header with only a number. In practice, it is used within an exercise or example as an alternative to \verb|itemize| or \verb|enumerate|.

    Answers can be requested within these environments using commands like \verb|\answer| or \verb|\wordChoice|. Hints, feedback, and solutions are also possible.

    \item \verb|hint|, \verb|feedback|, \verb|solution|:

    If there is a \verb|hint| within an exercise or example, an 'Hint' button appears online that reveals the content. Multiple hints are possible, and they can also contain, for example, \verb|\wordChoice|.

    The content of \verb|feedback| appears automatically once an answer is entered, and you can, for example, use \verb|\begin{feedback}[correct]| to display the feedback only for a correct answer.

    The \verb|solution| environment contains a worked-out solution that can be shown using a 'Solution' button. Beware: there is also an environment \verb|oplossing| that does \textit{not} have this functionality!

    By using \verb|\begin{solution}[show]|, the solution is always shown (also in handout mode). This is useful for examples, where the solution should be expandable online, but should be fully displayed in the handouts.

    Through settings, it can be determined whether hints/feedback/solutions should be displayed in the PDF or not.

    Technical detail: \verb|feedback| was internally misused to implement \verb|oplossing| (as \verb|\begin{feedback}{solution}|). However, as of 5/2023, there is an alternative and simpler implementation using pure CSS, via preamble.tex/global.css and .ximera/ximera.4ht, only in the summer course math/idl (and nowhere else at that time).

       
    \item \verb|onlineOnly|: only appears online (and in the full PDF in red)

    There is a counterpart, \verb|\pdfOnly|, which, however, for unclear reasons, is not implemented as an environment but as a command; see further.

    \item \verb|prompt|: appears online and in non-handout mode. Not used. 

    \item \verb|foldable| and \verb|expandable|: In HTML, this is collapsible or expandable; Ximera itself uses this functionality for hints and it is probably best NOT used directly. 

    \begin{foldable}
        CONTENT FOLDABLE

        Second line FOLDABLE.
    \end{foldable}

    \begin{expandable}
        CONTENT EXPANDABLE

        Second line EXPANDABLE.        
    \end{expandable} 

%     \item \verb|accordion| and \verb|xmuitweiding|: collapsible blocks.

% %     \begin{xmuitweiding}[title]   %TODO: makes compilation of xourse admin.tex fail !!!!
% %         CONTENT
% %     \end{xmuitweiding}

%      and for accordion:

%     %todo: move to printstyle 
%     \renewenvironment{accordion-item}[1][]{%
%        \par\textbf{#1}
%     }{%
%     }

    
%     \begin{accordion}
%     CONTENT ACCORDION
%         \begin{accordion-item}[title item 1]
%           CONTENT ITEM 1

%           Second line ITEM 1
%         \end{accordion-item}
%         \begin{accordion-item}[title item 2]
%      CONTENT ITEM 2
%         \end{accordion-item} 
%    \end{accordion}

    \item \verb|basicSkip| and \verb|basicOnly|: used to exclude certain elements from the PDF version. Extensively used for the A program of the Summer Science Course.

    basicSkip/basicOnly has no impact online (but it may be considered to turn basicSkip into a foldable/accordion in the online version)


\end{enumerate}


\end{document}